%
% cv.tex
%
% Author: Padraig O Conbhui
%
% Based heavily on resume.tex by
%   Jeremy B. R. Edberg <jedberg@gmail.com> http://www.jedberg.net
%


%****************************************************************************%


%
% Basic setup
%
\documentclass[11pt]{article}

\usepackage{fullpage}
\usepackage{setspace}

\pagestyle{empty}
\setlength{\tabcolsep}{0in}
\hyphenchar\font=-1

\usepackage{geometry} 
\geometry{a4paper, twoside} 
\setstretch{1}

\usepackage{booktabs}
\usepackage{topcapt}
\usepackage{tabulary}
\usepackage{hyphenat}


%*****************************************************************************%


%
% PDF setup
%
\usepackage{ifpdf}
\ifpdf
  \usepackage[pdftex]{hyperref}
\else
  \usepackage[hypertex]{hyperref}
\fi
\hypersetup{
    letterpaper,
    colorlinks,
    urlcolor=black,
    pdftitle={Resume - Padraig O Conbhui},
    pdfauthor={Padraig O Conbhui},
    pdfsubject={Resume - Padraig O Conbhui},
    pdfkeywords={%
        HPC, high performance computing, web development, physics, programming%
    }
}


%*****************************************************************************%


%
% Margin setup
%
\oddsidemargin  -0.5in
\evensidemargin -0.5in
\textwidth       7.2in
\headheight     -0.7in
\topmargin       0.0in
\textheight=     10.7in


%*****************************************************************************%


%
% Style setup
%
\usepackage{parskip}


%*****************************************************************************%


%
% Custom commands
%
\newcommand{\resumeSection}[1]{
    \par
    \vspace{\baselineskip}
    \large {\sc {#1}}
    \par
    \vspace{-0.9\baselineskip}
    \hrulefill
    \vspace{0.5\baselineskip}
    \par
}

\newenvironment{resumeSubSectionHeader}{
    \par
    \begin{tabular*}{\textwidth}{l@{\extracolsep{\fill}}r}
    \par
} {
    \end{tabular*}
    \par
}

\newenvironment{resumeSubSectionBody}{
    \par
    \vspace{-0.8\parskip}
    \begin{small}
    \par
} {
    \par
    \end{small}
    \par
}


%*****************************************************************************%


\begin{document}


%*****************************************************************************%


%
% Heading
%
\begin{center}
    { \huge \textbf \sc P\'{a}draig \'{O} Conbhu\'{\i} }

\begin{tabular*}{\textwidth}{@{\extracolsep{\fill}}lcr}
    poconbhui@gmail.com
    & &
    Mobile:
        +44 7598 421229
    \\
    \hline\hline
\end{tabular*}
\end{center}


%*****************************************************************************%


\resumeSection{Experience}


%
% University Of Edinburgh
%
\begin{resumeSubSectionHeader}
    \textbf{University Of Edinburgh}   & Edinburgh, Scotland \\
    \emph{PhD Student}                 & \emph{2013 - Present}

\end{resumeSubSectionHeader}
\begin{resumeSubSectionBody}

    Investigated numerical techniques for understanding how rocks record
    changes in the earth's magnetic field. This was done primarily through
    the lens of micromagnetic modelling.

    \begin{itemize}
        \item
            Developed a meshing program, MEshRRILL, using the CGAL library to
            generate tetrahedralized geometries of interest to us. This included
            simple polyhedra like cubes and truncated octahedra, but also
            realistic grain geometries reconstructed from FIB slice and view
            data. Meshes and surface reconstructions generated by this
            software have been used in publications.

        \item
            Developed software to produce simulated electron holography data
            from our micromagnetic models. This allowed us to directly compare
            experimental and simulation data for a given magnetic grain. This
            was wrapped up as a ParaView plugin and has been optimized to
            produce holography results in near real-time.
            Data generated by this software has been used in publications.

        \item
            Developed a computational technique for including magnetostrictive
            effects in micromagnetic models with free boundaries.
            The differential equations described by the problem are solved
            using the Finite Element Method, using the FEniCS environment
            to transform the equations into a matrix problem, and
            PETSc or Eigen to solve them.
            This was included in our group's serial micromagnetics code, MERRILL,
            and is compatible with our parallel code, DUNLOP.

        \item
            Developed workflows for compiling and distributing software so
            it could be reliably distributed as a standalone folder. This
            included static versions of programs like MERRILL, but also
            a shared version of MEshRRILL with dependencies installed in a
            subfolder, so it could easily be used as a C++ library to
            define complex geometries. The holography software was also compiled
            to work out of the box with a given ParaView binary downloaded
            from their website.
    \end{itemize}

\end{resumeSubSectionBody}

%
% WebSummit
%
\begin{resumeSubSectionHeader}

    \textbf{Dublin Web Summit Limited} & Dublin, Ireland   \\
    \emph{Technical Analyst}           & \emph{Summer 2012}

\end{resumeSubSectionHeader}
\begin{resumeSubSectionBody}

    Evaluated new technologies for use in projects.
    Built and maintained applications for internal and external use.
    Analyzed and modified applications to handle higher user loads
    as necessary.

    \begin{itemize}
        \item
            Developed browser based app leveraging Facebook's
            JavaScript API and FBQL to find users attending a Facebook event
            and list them in order of mutual friends with the current user,
            displaying those mutual friends, and showing which of them
            were also attending the event.
            This was tested against an event of about 3000 attendees and
            loaded results in several seconds.

        \item
            Analyzed ticketing system used for event, identified features
            causing the system to slow to an unacceptable pace when the
            user base grew large enough and presented solutions.

        \item
            Implemented caching and distribution of content through a CDN
            for the main site when hosting servers began straining under
            high traffic.

        \item
            Researched and made recommendations on IaaS platforms,
            deployments in cloud environments
            and NoSQL database systems for use with apps being built in
            house.
    \end{itemize}

\end{resumeSubSectionBody}


%
% LULI
%
\begin{resumeSubSectionHeader}

    \textbf{LULI, Ecole Polytechnique} & Palaiseau, France \\
    \emph{Intern}                      & \emph{Summer 2011}

\end{resumeSubSectionHeader}
\begin{resumeSubSectionBody}

    Investigated alignment errors found in experiments which used an
    ellipsoidal plasma mirror (EPM).

    \begin{itemize}
        \item
            Developed numerical model of an experimental setup testing
            an EPM using the Zemax optics software.

        \item
            Set up an equivalent physical model of the experimental setup
            using a Helium-Neon laser in a lab.

        \item
            Investigated alignment errors numerically due to geometric effects
            using ray tracing models, and compared them
            to diffractive errors by using wave front propagation models.

        \item
            Investigated alignment errors physically by playing with the
            lab setup.
    \end{itemize}

\end{resumeSubSectionBody}


%*****************************************************************************%


\resumeSection{Software Development}


%
% Software Development
%
\begin{resumeSubSectionBody}

    My preferred language is C\verb!++!,
    and my preferred application domain is anything I haven't done yet.
    Ideally, something that's never been done.
    I have a great interest in high performance programming
    and computational physics.
    I like to keep up to date with popular programming techniques and paradigms
    whenever possible.

    The technologies listed here are ones I would be happy to claim some
    expertise in.

    \begin{description}
        \item{\bf Projects:} \\
            \href{https://github.com/poconbhui}{github.com/poconbhui} \\
            \href{https://bitbucket.org/poconbhui}{bitbucket.org/poconbhui}

        \item{\bf Technologies used:} \\
            % Languages
            Bash, C, C\verb!++!, Fortran 90, Python,
            MPI, OpenMP, Cuda C, UPC, Coarray Fortran,
            \\
            % Libraries
            SWIG, FEniCS, CGAL, VTK, ParaView,
            Mathematica, MATLAB, Zemax, Gnuplot, LaTeX,
            \\
            PulseAudio, GStreamer, SBC, Bluetooth,
            \\
            % Web stuff
            HTML/CSS, JavaScript, PHP, Ruby,
            Ruby On Rails, Node.js, jQuery,
            \\
            % Databases
            SQL, CouchDB, MongoDB, Redis, Neo4j
    \end{description}

\end{resumeSubSectionBody}


%*****************************************************************************%


\resumeSection{Education \& Training}


%
% University Of Edinburgh
%
\begin{resumeSubSectionHeader}

    \textbf{University Of Edinburgh}       & Edinburgh, Scotland \\
    \emph{MSc, High Performance Computing} & \emph{2012 to 2013} \\
    \emph{Grade: MSc With Distinction}

\end{resumeSubSectionHeader}
\begin{resumeSubSectionBody}

    The course centred around building high performance applications,
    primarily through parallelisation, and through deep understanding
    of the underlying architecture of a given system and language.

    \begin{description}
        \item{\bf Subjects:}
            Message Passing Programming, Threaded Programming,
            Parallel Numerical Algorithms,
            Parallel Programming Languages, HPC Architectures,
            HPC Ecosystems, Performance Programming,
            Software Development, Advanced Parallel Programming,
            Parallel Design Patterns.
    \end{description}

\end{resumeSubSectionBody}


%
% Trinity College, Dublin
%
\begin{resumeSubSectionHeader}

    \textbf{Trinity College, Dublin}   & Dublin, Ireland     \\
    \emph{BA Mod, Theoretical Physics} & \emph{2008 to 2012} \\
    \emph{Grade: 2.1}

\end{resumeSubSectionHeader}
\begin{resumeSubSectionBody}

    The course taught problem solving and analytical thinking, providing
    tools necessary to break down and understand complex problems,
    primarily in the fields of Mathematics and Physics.
    A joint effort by both the School of Maths and the School of Physics,
    this course offered training as both a mathematician and a
    physicist.
    This training aimed primarily towards understanding condensed matter
    physics from the physics side, and quantum field theory and general
    relativity from the maths side.
    I completed a dissertation in the final year on performing particle
    physics calculations on GPUs using CUDA C.

    \begin{description}
        \item{\bf Subjects:}
            Computer Simulation, GPU Programming,
            Classical Mechanics, Quantum Mechanics,
            Classical Field Theory, Quantum Field Theory,
            Special Relativity, General Relativity,
            Classical Statistical Mechanics, Quantum Statistical Mechanics,
            Condensed Matter Physics, Spectroscopy,
            Optics, Electromagnetism, Thermodynamics,
            Electronics, Chaos and Complexity,
            Linear Algebra, Analysis,
            Topology.
    \end{description}

\end{resumeSubSectionBody}




%*****************************************************************************%


\resumeSection{Awards}


%
% Awards
%
\begin{resumeSubSectionBody}

    \begin{itemize}
        \setlength\itemsep{0em}
        \setlength\parskip{0em}
        \item
            % TODO: Find out actual name of this.
            Winner Castle Meeting 2016 Outstanding Participation.

        \item
            Creer Fund recipient 2016.

        \item
            Winner Geosciences PGR Conference 2013 Best Presentation.

        \item
            Included on TCD Dean of Students' Roll of Honour 2011.

        \item
            Three TCD book prizes for achieving 1st class honors grade in
            years 1-3.

        \item
            Awarded an Entrance Exhibition to Trinity College Dublin
            based on outstanding achievement in Leaving Certificate
            examination.

        \item
            Received full college scholarship ``Scol\'aireacht Neamhteoranta''
            for outstanding results in Irish language exams in the
            Leaving Certificate.

        \item
            Won battle of the bands, Col\'aiste Eoin, 2008 as singer and bassist
            of heavy metal band Capulus. \\
            (Demo available upon request)

        %\item
        %    I think Paddy is a pretty cool guy.
        %    Eh codes warez and doesn't afraid of anything.
    \end{itemize}

\end{resumeSubSectionBody}


%*****************************************************************************%


\end{document}
