%
% cv.tex
%
% Author: Paddy O Conbhui
%
% Based heavily on resume.tex by
%   Jeremy B. R. Edberg <jedberg@gmail.com> http://www.jedberg.net
%


%****************************************************************************%


%
% Basic setup
%
\documentclass[11pt]{article}

\usepackage{fullpage}
\usepackage{setspace}

\pagestyle{empty}
\setlength{\tabcolsep}{0in}
\hyphenchar\font=-1

\usepackage{geometry} 
\geometry{a4paper, twoside} 
\setstretch{1}

\usepackage{booktabs}
\usepackage{topcapt}
\usepackage{tabulary}
\usepackage{hyphenat}


%*****************************************************************************%


%
% PDF setup
%
\usepackage{ifpdf}
\ifpdf
  \usepackage[pdftex,bookmarks=true]{hyperref}
\else
  \usepackage[hypertex]{hyperref}
\fi
\hypersetup{
    letterpaper,
    colorlinks,
    urlcolor=black,
    pdftitle={CV - Paddy \'O Conbhu\'i},
    pdfauthor={Paddy O Conbhui},
    pdfsubject={CV - Paddy \'O Conbhu\'i},
    pdfkeywords={%
        HPC, high performance computing, web development, physics, programming%
    }
}


%*****************************************************************************%


%
% Margin setup
%
\oddsidemargin  -0.5in
\evensidemargin -0.5in
\textwidth       7.2in
\headheight     -0.7in
\topmargin       0.0in
\textheight=     10.7in


%*****************************************************************************%


%
% Style setup
%
\usepackage{parskip}


%*****************************************************************************%


%
% Custom commands
%
\newcommand{\resumeSection}[1]{
    \par
    \large {\sc {#1}}
    \par
    \vspace{-0.9\baselineskip}
    \hrulefill
    \vspace{0.25\baselineskip}
    \par
}

\newenvironment{resumeSubSectionHeader}{
    \par
    %\vspace{-0.15\baselineskip}
    \begin{tabular*}{\textwidth}{l@{\extracolsep{\fill}}r}
    \par
} {
    \end{tabular*}
    \par
}

\newenvironment{resumeSubSectionBody}{
    \par
    \vspace{-0.2\parskip}
    \begin{small}
    \par
} {
    \par
    \end{small}
    \par
}

\newenvironment{resumeItemize}{
    \vspace{-0.5\baselineskip}
    \begin{itemize}
} {
    \end{itemize}
}

\newenvironment{resumeDescription}{
    \vspace{-0.5\baselineskip}
    \begin{description}
} {
    \end{description}
}


%*****************************************************************************%


\begin{document}


%*****************************************************************************%


%
% Heading
%
\begin{center}
    { \huge \textbf \sc Paddy \'O Conbhu\'\i } \\
    poconbhui@gmail.com
\end{center}


%*****************************************************************************%


\resumeSection{Software Development}


%
% Software Development
%
\begin{resumeSubSectionBody}

    My preferred language is C\verb!++!, and my preferred application domain is
    anything I haven't done yet.
    Ideally, something that's never been done.
    I have a great interest in high performance computing, computational
    physics, and build systems.
    I like to keep up to date with modern programming techniques and paradigms
    whenever possible.

    All the technologies listed here I have used extensively in personal
    projects, or in a professional context.

    \begin{resumeDescription}
        \item{\bf Expertise:} \\
            % Languages
            Bash, C, C\verb!++!, Fortran 90, Python,
            \\
            MPI, OpenMP, CUDA C/C\verb!++!, CUDA Fortran,
            OpenACC, UPC, Coarray Fortran,
            \\
            % Compilers & build tools
            CMake, Autotools, GCC, Clang, MSVC, PGI, ELF/Mach-O linkers \& loaders,
            \\
            % Libraries
            SWIG, FEniCS, CGAL, VTK, ParaView,
            Mathematica, MATLAB, Zemax, Gnuplot, LaTeX,
            \\
            PulseAudio, GStreamer, SBC, Bluetooth,
            \\
            % Web stuff
            HTML/CSS, JavaScript, PHP, Ruby,
            Ruby On Rails, Node.js, jQuery,
            \\
            % Databases
            SQL, CouchDB, MongoDB, Redis, Neo4j
    \end{resumeDescription}

\end{resumeSubSectionBody}


%*****************************************************************************%


\resumeSection{Experience}


%
% Irish Centre for High-End Computing
%

\begin{resumeSubSectionHeader}

    \textbf{Irish Centre for High-End Computing (ICHEC)} & Dublin, Ireland \\
    \emph{Acting Oil \& Gas Programme Manager} & \emph{2018 - Present}

\end{resumeSubSectionHeader}
\begin{resumeSubSectionBody}

    Leading a research software engineering team of 9 post-docs with 1 PhD
    student with a focus on the oil \& gas domain,
    leading a long-term consultancy oil \& gas team of 5,
    and founded \& leading a consultancy Financial software optimization team of
    3.
    Lead and executed several bespoke AI performance optimization projects for
    Irish SMEs.

    \begin{resumeItemize}
        \item
            Managed the smooth transition of the duties and consultancy
            development work of departing Oil \& Gas Programme Manager to myself
            and the current consultancy oil \& gas team with limited disruption
            to our clients. Expanded and still expanding our capability and
            capacity to service oil \& gas clients.

        \item
            Founded a consultancy financial software optimization team. Won a
            highly competitive contract (competing with 4 other UK-based HPC
            consultancy firms) with a large financial software company for a
            (now completed) proof-of-concept consultancy optimization project.
            Further work with the client expected to begin in the next few
            months.

        \item
            Overseeing and contributing to the transition of the ExSeisDat
            project to its second phase focusing on hardware-level I/O
            optimization and exascale storage systems, securing ICHEC's
            partnerships with LERO and DDN to the end of the current funding
            cycle.

        \item
            Lead and contributed to an AI optimization \& predictive performance modelling project
            in the Machine Translation domain.

        \item
            Assessing and implementing internal policies and procedures to
            foster and develop critical skills and experience for our
            computational scientists.

        \item
            Developing best-practice guidelines, FAQs, and teaching material for
            software development within the centre.

        \item
            Developing capability within ICHEC to develop and present Nvidia GPU
            training material and workshops.
            Mentoring in Nvidia-sponsored OpenACC hackathons.
    \end{resumeItemize}
\end{resumeSubSectionBody}

\begin{resumeSubSectionHeader}

    \textbf{Irish Centre for High-End Computing (ICHEC)} & Dublin, Ireland \\
    \emph{Computational Scientist}                 & \emph{2017 - 2018}

\end{resumeSubSectionHeader}
\begin{resumeSubSectionBody}

    Contributed to the development of academic and enterprise software,
    primarily in the context of oil \& gas.
    Ported and optimized software for clients to leverage GPU accelerators,
    primarily for AI codes.

    \begin{resumeItemize}
        \item
            Contributed to the ExSeisDat library, focusing on parallel I/O
            and I/O optimization for oil \& gas data formats and workflows.
            Introduced a number of common best-practices, including using CMake,
            eliminating compiler warnings, using static analysis, automatic code
            formatting, and continuous integration using multiple compilers and
            dependency versions / implementations.

        \item
            Contributed to the development of consultancy oil \& gas software.

        \item
            Ported an oil \& gas code (Kirchhoff Depth Migration) to run on
            Microsoft Azure cloud platform, leveraging Nvidia GPU acceleration
            for a multi-national oil \& gas company.

        \item
            Analyzed and ported AI software stack in the Natural Language
            Processing domain to leverage GPU acceleration for Opening.io.

        \item
            Participated in numerous workshops (primarily with PRACE), further
            developing skills in high performance C\verb!++!, parallel I/O,
            seismic data processing, FPGA programming, performance analysis
            tools, and modern parallel programming models for exascale systems.

        \item
            Received professional certification as Certified Scrum Master (CSM)
            from Scrum Alliance.
    \end{resumeItemize}

\end{resumeSubSectionBody}


%
% University Of Edinburgh
%
\begin{resumeSubSectionHeader}

    \textbf{University Of Edinburgh}   & Edinburgh, Scotland \\
    \emph{PhD Student}                 & \emph{2013 - 2017}

\end{resumeSubSectionHeader}
\begin{resumeSubSectionBody}

    Investigated numerical techniques for understanding how rocks record
    changes in the earth's magnetic field. This was done primarily through
    the lens of micromagnetic modelling, using high-performance finite element
    analysis, C\verb!++! and Fortran.

    \begin{resumeItemize}
        \item
            Developed theory and computational techniques for including
            magnetostrictive effects in non-uniformly-magnetized 3d
            finite-element micromagnetic models with free boundaries.
            This was implemented in our group's micromagnetics code, MERRILL,
            with results presented at conferences, and published in a thesis.

        \item
            Developed a meshing application, MEshRRILL, leveraging the CGAL
            library to generate unstructured meshes for micromagnetic
            simulations.
            The application generates a number of pre-defined shapes, typical
            for ferromagnetic crystals, but also generates reconstructions of
            geometries from tomography images.
            Data generated by this software has been used in publications.

        \item
            Developed software, HoloMag, to produce simulated electron
            holography data from our micromagnetic models.
            This is provided as a ParaView plugin and has been optimized to
            produce holography results in near real-time, allowing researchers
            to develop an intuition for how 3d magnetizations effect the 2d
            holography images.
            Data generated by this software has been used in publications.

        \item
            Developed workflows for compiling software for reliable distribution.
            This included building as-static-as-possible versions of programs like MERRILL,
            along with shared-library version of MEshRRILL with all dependencies
            installed in a subfolder and RPATHs set appropriately, so it could
            easily be used and linked as a C++ library to mesh complex
            user-defined geometries.
            HoloMag was also compiled to be ABI-compatible with ParaView
            binaries downloaded from the ParaView website.
    \end{resumeItemize}

\end{resumeSubSectionBody}

%
% WebSummit
%
\begin{resumeSubSectionHeader}

    \textbf{Dublin Web Summit Limited} & Dublin, Ireland   \\
    \emph{Technical Analyst}           & \emph{Summer 2012}

\end{resumeSubSectionHeader}
\begin{resumeSubSectionBody}

    Evaluated new technologies for use in projects.
    Built and maintained applications for internal and external use.
    Analyzed and modified applications to handle higher user loads
    as necessary.

    \begin{resumeItemize}
        \item
            Developed browser based app leveraging Facebook's JavaScript API and
            FBQL API to display a user's mutual friends with attendees of an
            event.
            Handled events with thousands of attendees within seconds.

        \item
            Analyzed ticketing system used for event, identified features
            causing the system to slow to an unacceptable pace when the
            user base grew large enough and presented solutions.

        \item
            Implemented caching and distribution of content through a CDN
            for events sites when hosting servers began straining under
            high traffic.

        \item
            Researched and made recommendations on IaaS platforms,
            deployments in cloud environments
            and NoSQL database systems for use with apps and services being
            developed in-house.
    \end{resumeItemize}

\end{resumeSubSectionBody}


%
% LULI
%
\begin{resumeSubSectionHeader}

    \textbf{LULI, \'Ecole Polytechnique} & Palaiseau, France \\
    \emph{Intern}                      & \emph{Summer 2011}

\end{resumeSubSectionHeader}
\begin{resumeSubSectionBody}

    Investigated alignment errors encountered by experiments using
    ellipsoidal plasma mirrors (EPM) for focusing lasers.

    \begin{resumeItemize}
        \item
            Participated in a laser-plasma ion acceleration experiment.

        \item
            Developed numerical model of an experimental setup probing EPM
            alignment errors using the Zemax optics software.

        \item
            Investigated alignment errors due to geometric effects using ray
            tracing models, and compared them to diffractive errors by using
            wave front propagation models.

        \item
            Set up an equivalent physical model of the experimental setup using
            a Helium-Neon laser to experimentally replicate the predictions of
            the model.

        \item
            Demonstrated the classes of alignment errors the EPM was and was not
            sensitive to, informing alignment procedures in future experiments.
    \end{resumeItemize}

\end{resumeSubSectionBody}


%*****************************************************************************%


\pagebreak
\resumeSection{Formal Education \& Training}


%
% University Of Edinburgh
%
\begin{resumeSubSectionHeader}

    \textbf{University Of Edinburgh}       & Edinburgh, Scotland \\
    \emph{PhD, Geophysics} & \emph{2013 to 2017} \\

\end{resumeSubSectionHeader}
\begin{resumeSubSectionBody}
    Trained as a researcher in Geophysics,
    primarily in computational paleomagnetism.
    \begin{resumeDescription}
        \item{
            {\bf Thesis:}
            \it Micromagnetic Modelling of Imperfect Crystals
        } \\
        Developed the theory needed to simulate non-uniformly-magnetized,
        multi-phase, ferromagnetic crystals including magnetostriction and
        crystal defects, with free boundary conditions.
        Presented implementations and results.
    \end{resumeDescription}
\end{resumeSubSectionBody}

\begin{resumeSubSectionHeader}

    \textbf{University Of Edinburgh}       & Edinburgh, Scotland \\
    \emph{MSc, High Performance Computing -- With Distinction} & \emph{2012 to 2013}

\end{resumeSubSectionHeader}
\begin{resumeSubSectionBody}

    The course centred around building high performance applications,
    primarily through parallelism, and through deep understanding
    of the underlying architecture of a given system and language.

    \begin{resumeDescription}
        \item{\bf Subjects:}
            Message Passing Programming, Threaded Programming,
            Parallel Numerical Algorithms,
            Parallel Programming Languages, HPC Architectures,
            HPC Ecosystems, Performance Programming,
            Software Development, Advanced Parallel Programming,
            Parallel Design Patterns.

        \item{
            {\bf Dissertation:}
            \it Towards Exascale Molecular Dynamics
        } \\
            Compared replicated data and systolic loop schemes for
            molecular dynamics simulations, and presented a novel hybrid
            replicated systolic scheme with impressive scaling properties.
    \end{resumeDescription}

\end{resumeSubSectionBody}


%
% Trinity College, Dublin
%
\begin{resumeSubSectionHeader}

    \textbf{Trinity College, Dublin}   & Dublin, Ireland     \\
    \emph{BA Mod, Theoretical Physics -- II-1} & \emph{2008 to 2012}

\end{resumeSubSectionHeader}
\begin{resumeSubSectionBody}

    The course taught practical problem solving skills and analytical thinking
    in the fields of Mathematics and Physics.

    \begin{resumeDescription}
        \item{\bf Subjects:}
            Computer Simulation, GPU Programming,
            Experimental Labs,
            Classical Mechanics, Quantum Mechanics,
            Classical Field Theory, Quantum Field Theory,
            Special Relativity, General Relativity,
            Classical Statistical Mechanics, Quantum Statistical Mechanics,
            Condensed Matter Physics, Spectroscopy,
            Optics, Electromagnetism, Magnetism, Thermodynamics,
            Electronics, Chaos and Complexity,
            Linear Algebra, Analysis,
            Topology.

        \item{
            {\bf Dissertation:}
            \it Computing Two Point Correlators For A Lattice QCD Theory On
            Graphics Processing Units
        } \\
            Wrote complex-valued matrix multiplications routines in CUDA C which
            beat cuBLAS (at the time) by 20\%.
            Re-cast the full set of matrix multiply-traces needed to compute the
            correlator matrix into a large matrix-matrix multiplication,
            providing up to 20x performance improvement over independent
            multiply-trace operations.
    \end{resumeDescription}

\end{resumeSubSectionBody}




%*****************************************************************************%


% \resumeSection{Awards}
% 
% 
% %
% % Awards
% %
% \begin{resumeSubSectionBody}
% 
%     \begin{resumeItemize}
%         \setlength\itemsep{0em}
%         \setlength\parskip{0em}
%         \item
%             % TODO: Find out actual name of this.
%             Winner Castle Meeting 2016 Outstanding Participation.
% 
%         \item
%             Creer Fund recipient 2016.
% 
%         \item
%             Winner Geosciences PGR Conference 2013 Best Presentation.
% 
%         \item
%             Included on TCD Dean of Students' Roll of Honour 2011.
% 
%         \item
%             Three TCD book prizes for achieving 1st class honors grade in
%             years 1-3.
% 
%         \item
%             Awarded an Entrance Exhibition to Trinity College Dublin
%             based on outstanding achievement in Leaving Certificate
%             examination.
% 
%         \item
%             Received full college scholarship ``Scol\'aireacht Neamhteoranta''
%             for outstanding results in Irish language exams in the
%             Leaving Certificate.
% 
%         \item
%             Won battle of the bands, Col\'aiste Eoin, 2008 as singer and bassist
%             of heavy metal band Capulus. \\
%             (Demo available upon request)
% 
%         %\item
%         %    I think Paddy is a pretty cool guy.
%         %    Eh codes warez and doesn't afraid of anything.
%     \end{resumeItemize}
% 
% \end{resumeSubSectionBody}



%*****************************************************************************%


\resumeSection{Publications}


%
% Published
%
\begin{resumeSubSectionBody}

    Fisher, M. A., Conbhuí, P. Ó., ... {\&} Short, R. (2018). ExSeisDat: A set of parallel I/O and
    workflow libraries for petroleum seismology. Oil {\&} Gas Science and
    Technology–Revue d’IFP Energies nouvelles, 73, 74.

    Ó Conbhuí, P. (2018). Micromagnetic modelling of imperfect crystals, PhD
    Thesis, University of Edinburgh.

    Conbhuí, P. Ó., Williams, W., Fabian, K., Ridley, P., Nagy, L., {\&}
    Muxworthy, A. R. (2018). MERRILL: Micromagnetic earth related robust
    interpreted language laboratory. Geochemistry, Geophysics, Geosystems,
    19(4), 1080-1106.

    Valdez-Grijalva, M. A., Muxworthy, A. R., Williams, W., Conbhuí, P. Ó.,
    Nagy, L., Roberts, A. P., {\&} Heslop, D. (2018). Magnetic vortex effects on
    first-order reversal curve (FORC) diagrams for greigite dispersions. Earth
    and Planetary Science Letters, 501, 103-111.

    Nagy, L., Williams, W., Muxworthy, A. R., Fabian, K., Almeida, T. P.,
    Conbhuí, P. Ó., {\&} Shcherbakov, V. P. (2017). Stability of equidimensional
    pseudo–single-domain magnetite over billion-year timescales. Proceedings of
    the National Academy of Sciences, 114(39), 10356-10360.

    Almeida, T. P., Muxworthy, A. R., Kov\'acs, A., Williams, W., Nagy, L.,
    Conbhu\'i, P. \'O., ... {\&} Dunin-Borkowski, R. E. (2016). Direct
    observation of the thermal demagnetization of magnetic vortex structures in
    nonideal magnetite recorders. Geophysical Research Letters, 43(16),
    8426-8434.

    Einsle, J. F., Harrison, R. J., Kasama, T., Conbhu\'i, P. \'O., Fabian, K.,
    Williams, W., ... {\&} Midgley, P. A. (2016). Multi-scale three-dimensional
    characterization of iron particles in dusty olivine: Implications for
    paleomagnetism of chondritic meteorites. American Mineralogist, 101(9),
    2070-2084.

    Einsle, J. F., Fu, R. R., Weiss, B. P., Kasama, T., Fabian, K., Conbhu\'i,
    P.  \'O., ... {\&} Harrison, R. J. (2015). Focused ion beam nanotomography
    of chondritic meteorites: Closing the mesoscale length gap in paleomagnetic
    studies. Microscopy and Microanalysis, 21(S3), 2261-2262.

\end{resumeSubSectionBody}


%*****************************************************************************%


\end{document}
